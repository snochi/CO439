\documentclass[11pt]{article}

%% ========================================================
%% package imports

\usepackage{infimum}

%% ========================================================
%% document


\begin{document}

    
    \section{Polynomial Rings and Ideals}

    \subsection{Preliminaries}

    \begin{definition}{\textbf{Algebraically Closed} Field}
        We say a field $K$ is \emph{algebraically closed} if 
    \end{definition}

    \begin{definition}{\textbf{Degree}, \textbf{Support} of a Polynomial}
        Let $f = \sum^{}_{i}c_{vec}\vec{x}^{\vec{a}}$ be a polynomial. Then the \textbf{degree} of $f$ is
        \begin{equation*}
            \deg\left( f \right) = \phantom{\max\left\lbrace \sum^{}_{i}a_i: c_{\vec{a}}\neq 0 \right\rbrace}
        \end{equation*}
        The \emph{support} of $f$ is
        \begin{equation*}
            \supp\left( f \right) = \phantom{\left\lbrace \vec{a}: c_{\vec{a}}\neq 0 \right\rbrace}
        \end{equation*}
    \end{definition}

    \begin{definition}{\textbf{Homogeneous} Polynomial}
        We say $f$ is \textbf{homogeneous} of degree $i$ if 
    \end{definition}

    \begin{definition}{\textbf{Hilbert Series} of a Graded Ring}
        A \emph{Hilbert series} of a graded ring $S$ is
        \begin{equation*}
            H\left( S;t \right) = \phantom{\sum^{}_{i\in\N} \dim\left( S_i \right) t^i}
        \end{equation*}
    \end{definition}

    \begin{definition}{\textbf{Ideal} of a Polynomial Ring}
        An \emph{ideal} is 
        \begin{equation*}
        \end{equation*}
    \end{definition}
    
    \begin{definition}{Ideal \textbf{Generated} by a Subset}
        Let $F\subseteq K\left[ \vec{x} \right]$. We define the ideal \emph{generated} by $F$, denoted as $\left< F \right>$ (or $\left( F \right)$), to be
        \begin{equation*}
            \left< F \right> = \phantom{\bigcap^{}_{} \left\lbrace I\supseteq F: \text{$I$ is an ideal} \right\rbrace}
        \end{equation*}
        or 
    \end{definition}
    
    \begin{definition}{\textbf{Homogeneous} Ideal}
        An ideal $I\subseteq K\left[ \vec{x} \right]$ is \emph{homogeneous} if 
    \end{definition}
    
    \begin{definition}{\textbf{Quotient Ring}}
        Let $I\subseteq K\left[ \vec{x} \right]$ be an ideal. Given any $f\in K\left[ \vec{x} \right]$, the \emph{residue class} of $f$ modulo $I$ is
        \begin{equation*}
            \phantom{f+I = \left\lbrace f+i: i\in I \right\rbrace}
        \end{equation*}
        Then there is an equivalence relation $\sim$ on $K\left[ \vec{x} \right]$ by
        \begin{equation*}
            \phantom{f\sim g \iff f+I = g+I}
        \end{equation*}
        The \emph{quotient ring} $S /I$ is the ring structure on the set of all residue classes $S /\sim$, with addition
        \begin{equation*}
            \phantom{\left( f+I \right) + \left( g+I \right) = \left( f+g \right) + I}
        \end{equation*}
        and multiplication
        \begin{equation*}
            \phantom{\left( f+I \right)\left( g+I \right) = fg+I}
        \end{equation*}
    \end{definition}
    
    \begin{definition}{\textbf{Product Ideal}, \textbf{Colon Ideal} of Two Ideals}
        Let $I,J$ be ideals. We define the \emph{product ideal} of $I,J$, denoted as $IJ$, to be
        \begin{equation*}
            IJ = \phantom{\left< fg: f\in I, g\in J \right>}
        \end{equation*}

        We also define the \emph{colon ideal} of $I,J$, denoted as $I:J$, as
        \begin{equation*}
            I:J = \phantom{\left\lbrace f\in S: \forall j\in J\left[ fj\in I \right] \right\rbrace = \left\lbrace f\in S: fJ\subseteq I \right\rbrace}
        \end{equation*}
    \end{definition}
    
    \begin{definition}{\textbf{Radical} of an Ideal}
        Let $I$ be an ideal. The \emph{radical} of $I$ is
        \begin{equation*}
            \sqrt{I} = \phantom{\left\lbrace f\in S: \exists k\in\N\left[ f^k\in I \right] \right\rbrace}
        \end{equation*}
    \end{definition}

    \section{Algebraic Geometry}

    \subsection{Introduction}

    \begin{definition}{\textbf{Vanishing Locus} of a Subset of a Polynomial Ring}
        For any $F\subseteq S$, we define a \emph{variety} $V\left( F \right)$ (or $V_K\left( F \right)$ when we want to specify the field $K$) by
        \begin{equation*}
            V\left( F \right) = \phantom{\left\lbrace \vec{p}\in K^n: \forall f\in F\left[ f\left( \vec{p} \right) = 0 \right] \right\rbrace}
        \end{equation*}
        called the \emph{vanishing locus} of $F$.
    \end{definition}
    
    \subsection{Schemes}
    
    \begin{definition}{\textbf{Radical Ideal} of a Variety}
        Let $X = V\left( F \right)$ be a variety. Then the \emph{radical ideal} of $X$ is the set
        \begin{equation*}
            I\left( X \right) = \phantom{\left\lbrace f\in K\left[ \vec{x} \right] : \forall x\in X\left[ f\left( x \right)=0 \right] \right\rbrace}
        \end{equation*}
    \end{definition}

    \begin{definition}{\textbf{Scheme} of a Set of Polynomials}
        For every $F\subseteq K\left[ \vec{x} \right]$, we define the \emph{scheme} of $F$, denoted as $V_{\infty}\left( F \right)$, by
        \begin{equation*}
            V_{\infty}\left( F \right) = \phantom{\left\lbrace V_R\left( F \right) : \text{$R\supseteq K$ is a ring extension of $K$} \right\rbrace}
        \end{equation*}
    \end{definition}

    \subsection{Monomial Ideals}
    
    \begin{definition}{\textbf{Monomial Ideal}}
        A \emph{monomial ideal} in $K\left[ \vec{x} \right]$ is 
    \end{definition}

    \begin{definition}{\textbf{Hyperplane} of a Vector Space}
        A \emph{hyperplane} is 

        We say a hyperplane is \emph{coordinate} if 
    \end{definition}
    
    \begin{definition}{\textbf{Minimal} Set of Monomials}
        Let $M$ be a set of monomials. We say $M$ is \emph{minimal} if 
    \end{definition}
    
    \begin{definition}{\textbf{Canonical Generating Set} of a Monomial Ideal}
        Let $I$ be a monomial ideal. The \emph{canonical generating set} of $I$, denoted as $G\left( I \right)$, is
    \end{definition}
    
    \subsection{Operations on Monomial Ideals}
    
    \subsection{Grobner Bases}

    \begin{definition}{\textbf{Monomial Order}}
        A \emph{monomial order} is a total order on $\N^n$ such that
        \begin{enumerate}
            \item \hfill\textit{shifting}
            \item
        \end{enumerate}
    \end{definition}

    \begin{definition}{\textbf{Initial Monomial} of a Polynomial}
        Let $S = K\left[ \vec{x} \right]$ and fix a monomial order $<$. For $f\in S$, if $f$ is nonzero, we define the \emph{initial monomial} (or \emph{leading monomial}) of $f$, denoted as $\initial_<\left( f \right)$, as
        \begin{equation*}
            \initial_< \left( f \right) = \phantom{\text{$<$-greatest monomial of $\supp\left( f \right)$}}
        \end{equation*}
        In case $f=0$, we set
        \begin{equation*}
            \initial_<\left( f \right)=\phantom{0}
        \end{equation*}

        The \emph{leading coefficient} (or \emph{initial coefficient}) is

        The \emph{leading term} (or \emph{initial term})is
    \end{definition}

    \begin{definition}{\textbf{Initial Ideal} of an Ideal}
        Let $I$ be an ideal. We define the \emph{initial ideal} of $I$, denoted as $\initial_<\left( I \right)$, by
        \begin{equation*}
            \initial_<\left( I \right) = \phantom{\left< \initial_<\left( f \right):f\in I \right>}
        \end{equation*}
    \end{definition}

    \begin{definition}{\textbf{Grobner Basis} for an Ideal}
        We say $G=\left\lbrace g_i \right\rbrace^{k}_{i=1}$ is a \emph{Grobner basis} for an ideal $I$ if
        \begin{equation*}
            \phantom{\initial\left( I \right) = \left< \initial\left( g_i \right) \right>^k_{i=1}}
        \end{equation*}
    \end{definition}

    \subsection{Division Algorithm}
    
    \subsection{Buchberger's Algorithm for Finding a Grobner Basis}

    \begin{definition}{\textbf{$S$-polynomial}}
        Fix a monomial order. For all $f,g\in K\left[ x \right]$, the \emph{$S$-polynomial} of $f$ by $g$, denoted as $S\left( f,g \right)$, is defined as
        \begin{equation*}
            S\left( f,g \right) = \phantom{\frac{\lcm\left( \initial\left( f \right),\initial\left( g \right) \right)}{c\initial\left( f \right)} f - \frac{\lcm\left( \initial\left( f \right),\initial\left( g \right) \right)}{d\initial\left( g \right)}g}
        \end{equation*}
    \end{definition}
    
    \begin{definition}{\textbf{Reduces to $0$} Modulo $g_1,\ldots,g_m$}
        We say $f\in K\left[ \vec{x} \right]$ \emph{reduces to $0$} modulo $g_1,\ldots,g_m$ if
        \begin{equation*}
            \phantom{f = \sum^{m}_{i=1} q_ig_i}
        \end{equation*}
    \end{definition}

    \subsection{Reduced Grobner Bases}
    
    \begin{definition}{\textbf{Reduced} Grobner Basis}
        Let $G = \left\lbrace g_k \right\rbrace^{m}_{j=1}$ be a Grobner basis for an ideal $I$. We say $G$ is \emph{reduced} if
        \begin{enumerate}
            \item 
            \item 
        \end{enumerate}
    \end{definition} 
    



\end{document}
